\chapter{Approach}\label{chap:approach}
This section is a point by point depiction of the actualized framework utilized in this thesis, to detect the Tweets related to Human Migration  and its Sentiment. To start with, the undertaking of Mining Twitter data is defined, and next the framework is depicted in the sequential order of its working.

\section{Problem Formulation}

Social media generates a massive amount of data every minute, Which has helped and created 
many business opportunities. These data can be mined to find valuable patterns. Data regarding human mobility or migration flow can be used for many applications such as urban planning 
or understanding disease spreading or even population prediction. But, data collection method
 is outdated as data scientist use census data for identifying the patterns in inflow and outflow 
of humans, a lot of data and prediction is lost between census. Recent studies have shown that
this trends or patterns can be studied using tracking mobile phones signals, GPS, WIFI or even
with RFID devices. But, with these techniques, there will be privacy concerns and data access 
restriction. 

With focus on mining twitter data and gathering information on specific topic which is migration in
my paper. I need to understand why people migrate. And implement a strategy to mine migration
tweets. This data mining strategy has to be evaluated. Once the Tweets are classified as migration
tweets, these tweets are used to understand the sentiment of the people who are tweeting this. So
the work in my thesis concentrate on following questions.
1. How is data related to human migration mined from twitter? What are the key
parameters to filter tweets regarding human migration from other generic subject Tweets?
2. Sentiment detection based on twitter migration data between TWO countries?


\section{Method}