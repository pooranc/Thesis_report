\chapter{Background}\label{chap:background}
This chapter displays the background of the investigation, to enable a better comprehension of the framework executed. It starts with an introduction to the task of Detecting the sentiment of tweets related to human migration, followed by Text mining, Details of Twitter's data representation and its API's, classification algorithms. followed by the architectures that take advantage of the data’s spatial and temporal properties. The accompanying segment focuses on the past work that has been done in the field of mining Twitter's information and its analysis.


    
\section{Technical Background}

The core task of detecting the Sentiment of the Tweets which are related to "Human Migration" requires techniques related to Data mining, Text mining and Machine Learning. For instance, typical computational analysis systems first extract Data regarding the "migration" from the tweets, and supervised classifier such as Logistic regression or artificial neural network (ANNs) are used for classification and detection. Thus, Data mining and classifier algorithms are the core tools in this computation analysis and hence are covered in this section.

\subsection{Twitter Data mining}
\subsection{Text mining}
\subsection{Classification}
\subsection{Twitter API's and Data representation}


Social media platform can be used to collect data. Twitter is one of the well known online social networks, Where people use this to communicate short messages called tweets. Posting short messages which is called Tweeting and express opinion on various topics that they are interested in. Another description of Twitter and tweeting might be microblogging. \cite{TwitterDevDocs} Gives a
complete list of all Twitter API. Which can be used to mine data from Twitter and filter data
accordingly to match our research. 

\underline{matte heavy agi bari}
1) twitter data sets bagge bari
2) Twitter api bagge bari
3) data mining with tweets bagge bari



\\ \cite{Marco} Gives in-detail about how the data in Twitter are represented
and how the data can be extracted. Twitter contains Hashtags, Topics and Time Series along with
Users, Followers, and Communities. And  \cite{Marco} also explains how to filter hashtags, represent the
tweets in graphs. Tweets are spatial-temporal data as these contain Geo-tag along with Created
Time metadata.

\section{Related works in mining Twitter social media data}



\\
\cite{Goergen:2014:STA:2670386.2670392} paper concentrates on event detection, which is mainly focused on 3 distinct parameters
known as w3 questions, "What is really happening, Where is the incident and When did the
event happen?". The authors collects the spatial-temporal data from Twitter from two different
countries. They use Names entity recognition, Geo-coordinates to identify location. Using this
data authors determine the "number of possible users for a shared account by calculating the
distance and velocity between tweets belonging to single account".

\\ \cite{Hübl} paper concentrates on "Individual and aggregate trajectories that reflects the refugee migration
movements" and "identify the spatiotemporal event clusters of refugee-related tweets to
likely determine the location of refugees". The technique used by authors to collect data is, they
downloaded all the Twitter data in a specific time frame and apply filters to separate out tweets
which were of geographic interest. Authors collect both tweets which were geotagged with coordinates
and with geotagged with a place description. For the latter type, The twitter places were
are extended over continents. In addition to above filtering method, the authors also filter tweets
of interest based on hashtag search. \cite{Hübl} work inspired me on how to collect word list for filtering
tweets related to migration. But this work was related to the refugee crisis.
 

\\ Compare to the data collection method used by \cite{Hübl}. The authors of \cite{Cortis:2015:ACT:2809563.2809605} use hashtag filtering
along with popular terms used in those tweets. \cite{Cortis:2015:ACT:2809563.2809605} tries to analyze cyberbullying tweets in trending
world events. The authors choose two world events which were the cause of several cyberbullying.
The technique used by authors to collect and filter Twitter data is that they select two real-world
events trending hashtags along with popular cyberbullying terms. This technique inspired me to
apply this filter logic to collect tweets regarding migration.
Regarding Sentiment analysis there are two approaches, One approach is to use the labeled
texts and use supervised machine learning trained on the labeled text data to classify the polarity
of new texts. Another approach creates a sentiment lexicon and scores the text based on some
function that describes how the words and phrases of the text match the lexicon, \cite{DBLP:journals/corr/abs-1103-2903} Evaluates
word list sentiment analysis for microblogging.

\\ \cite{Jamie} Discusses how a classifier model is trained
and evaluated by calculating precision, recall and F1 score on manually annotated and classifier
predictions.


\\ \underline{matte heavy agi bari}
