\chapter{Introduction}\label{chap:introduction}
Migration is one of the social phenomena in human beings life. It is the physical movement of human beings from one place to another for short or long term within or state borders. The sociologist has come up with the push-pull model after lots of research. This model tells or compares the push factors, that usually happens when sending country forces people to leave their homes, to pull factors, which attracts the people onto receiving country. In the recent history, the reasons for migration has changed and its changing. Today people migrate as an asylum seeker or as a trafficked person or refugee or as an international student.

Data Regarding migration flows are largely inconsistent across countries, typically outdated and
are often nonexistent. Research on migration policy making from International Organization for
Migration(IOM)  \cite{IOM's} quotes "Migration has become one of the most challenging issues confronting
policymakers around the world. The growing complexity of internal and cross-border human mobility
has highlighted the need for reliable and timely data to inform migration policy development
and humanitarian assistance". 

Migration data is one the important parameter for demographic changes, But the traditional methods of collecting and storing data of migration statistics are limited. The census which is the comprehensive source of information about the entire population usually takes place every 5 to 10 years, some data is lost between census cycles. Methods like census and population registers may often not be appropriate to estimate short-term migration. 

Spread of internet application, social media, and mobile phones as massively effected on how we communicate with others. This has helped many researchers to study on societies and due to the Digital revolution, these data are stored. The key feature in these data are Time and Geo-location. Availability of Spatial-Temporal data from on-line sources have opened up new opportunities for data scientist. These data can be accessed by anyone with relevant technical skills like Twitter data-sets, Linked-In data-sets, mobile phone data or can be bought from the companies like Google and Facebook (they don't sell the user data). Mainly, the data collected is based on the data desired. "Time" would be one of the important parameter to estimate the migration trends. Along with that, a specific group of population data (example: "Refugee") could be used to analyze trends. Human migration flow based on political reason, short-term migration for Education, employment and gender can be used to collect data. But obtaining unbiased data for all the topics would be difficult. 

A successful approach to identify Tweets which are related to cyberbullying has been proposed by \cite{Cortis}. My approach differs from the mentioned approach, as I try to classify the Tweets which are related to Migration. I classify on the level of single tweets (\underline{change maadu}), which makes it applicable for classifying based on the twitter streaming API, and identifying Migration Tweets from other generic topic tweet. With focus on mining twitter data and gathering information on specific topic which is Human migration in my paper. I need to understand why people migrate. And implement a strategy to mine migration tweets. Once the Tweets are classified as migration tweets, these tweets are used to understand the sentiment of the people who are tweeting this. So the work in my thesis concentrate on following questions. 

\begin{itemize}
  \item A framework to classify Tweets which are related to human migration.
  \item Sentiment detection based on twitter migration data between TWO countries? 
\end{itemize}

This paper is organized as follows: In Section 2 we review
related work and in Section 3 we describe our classification
approach and its concrete implementation. The results are
presented in Section 4, accompanied by an empirical analysis of computer scientists on Twitter in Section 5. We draw
conclusions about our approach in Section 6.


In this 


\underline{brief agi bari about the approach}

