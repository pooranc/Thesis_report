\chapter{Introduction}\label{chap:introduction}
Migration is one of the social phenomena in human beings life. It is the physical movement of human beings from one place to another for short or long term within or state borders. The sociologist has come up with the push-pull model after lots of research. This model tells or compares the push factors, that usually happens when sending country forces people to leave their homes, to pull factors, which attracts the people onto receiving country. In the recent history, the reasons for migration has changed and its changing. Today people migrate as an asylum seeker or as a trafficked person or refugee or as an international student.

\section{Example citation \& symbol reference}\label{sec:citation}
For symbols look at \cite{latex_symbols_2017}.


\section{Example reference}
Example reference: Look at chapter~\ref{chap:introduction}, for sections, look at section~\ref{sec:citation}.

\section{Example image}

\begin{figure}
	\centering
	\includegraphics[width=0.5\linewidth]{uni-logo}
	\caption{Meaningful caption for this image}
	\label{fig:uniLogo}
\end{figure}

Example figure reference: Look at Figure~\ref{fig:uniLogo} to see an image. It can be \texttt{jpg}, \texttt{png}, or best: \texttt{pdf} (if vector graphic).

\section{Example table}

\begin{table}
	\centering
	\begin{tabular}{lr}
		First column & Number column \\
		\hline
		Accuracy & 0.532 \\
		F1 score & 0.87
	\end{tabular}
	\caption{Meaningful caption for this table}
	\label{tab:result}
\end{table}





Table~\ref{tab:result} shows a simple table\footnote{Check \url{https://en.wikibooks.org/wiki/LaTeX/Tables} on syntax}